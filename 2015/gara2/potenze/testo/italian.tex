% \usepackage{xcolor}
% \usepackage{afterpage}
% \usepackage{pifont,mdframed}
% \usepackage[bottom]{footmisc}

\makeatletter
\gdef\this@inputfilename{input.txt}
\gdef\this@outputfilename{output.txt}
\makeatother

\newcommand{\inputfile}{\texttt{input.txt}}
\newcommand{\outputfile}{\texttt{output.txt}}

\newenvironment{warning}
  {\par\begin{mdframed}[linewidth=2pt,linecolor=gray]%
    \begin{list}{}{\leftmargin=1cm
                   \labelwidth=\leftmargin}\item[\Large\ding{43}]}
  {\end{list}\end{mdframed}\par}

	Gabriele si è recentemente appassionato al linguaggio C$--$ (in particolare, C$--$11), in cui le uniche variabili ammesse sono array $D$-dimensionali con tutte le dimensioni uguali tra loro $M$ (una variabile siffatta conterrà quindi $M^D$ celle di memoria). Una delle novità introdotte dal C$--$11 rispetto al C$--$99, è che d'ora in poi sarà ammesso dichiarare un'unica variabile, che inoltre deve avere dimensione $D$ almeno pari a 2.

	Gabriele sta aggiornando il suo vecchio codice ai nuovi standard, e sta riscontrando qualche difficoltà con la gestione della memoria in alcuni dei suoi programmi più complessi. Sapendo che il suo computer ha una capacità di $N$ celle di memoria, quante ne potrà al massimo sfruttare con un programma di C$--$11 (e quindi con un'unica variabile di dimensione una potenza $M^D$ con esponente almeno 2)?


\Implementation
Dovrai sottoporre esattamente un file con estensione \texttt{.c}, \texttt{.cpp} o \texttt{.pas}.

\begin{warning}
Tra gli allegati a questo task troverai un template (\texttt{potenze.c}, \texttt{potenze.cpp}, \texttt{potenze.pas}) con un esempio di implementazione da completare.
\end{warning}

Se sceglierai di utilizzare il template, dovrai implementare la seguente funzione:
\begin{center}\begin{tabularx}{\textwidth}{|c|X|}
\hline
C/C++  & \verb|int alloca(int N);|\\
\hline
Pascal & \verb|function alloca(N: longint): longint;|\\
\hline
\end{tabularx}\end{center}
In cui:
\begin{itemize}[nolistsep]
  \item L'intero $N$ rappresenta il numero di celle di memoria presenti nel computer di Gabriele.
  \item La funzione dovrà restituire il massimo numero di celle allocabile in C$--$11 (e quindi la più grande potenza $M^D$ minore o uguale a $N$ con $D \ge 2$), che verrà stampato sul file di output.
\end{itemize}

\InputFile
Il file \inputfile{} è composto da un'unica riga contenente l'unico intero $N$.

\OutputFile
Il file \outputfile{} è composto da un'unica riga contenente un unico intero, la risposta a questo problema.

% Assunzioni
\Constraints
\begin{itemize}[nolistsep, itemsep=2mm]
	\item $1 \le N \le 100\,000$.
\end{itemize}

\pagebreak
\Scoring
Il tuo programma verrà testato su diversi test case raggruppati in subtask.
Per ottenere il punteggio relativo ad un subtask, è necessario risolvere
correttamente tutti i test relativi ad esso.

\begin{itemize}[nolistsep,itemsep=2mm]
  \item \textbf{\makebox[2cm][l]{Subtask 1} [10 punti]}: Casi d'esempio.
  \item \textbf{\makebox[2cm][l]{Subtask 2} [20 punti]}: $N \leq 10$.
  \item \textbf{\makebox[2cm][l]{Subtask 3} [40 punti]}: $N \leq 1000$.
  \item \textbf{\makebox[2cm][l]{Subtask 4} [30 punti]}: Nessuna limitazione specifica.
\end{itemize}

% Esempi
\Examples
\begin{example}
\exmp{
10
}{%
9
}%
\end{example}
\begin{example}
\exmp{
32
}{%
32
}%
\end{example}


\Explanation
Nel \textbf{primo caso di esempio}, la potenza più grande minore o uguale a 10 è $9 = 3^2$.\\[2mm]
Nel \textbf{secondo caso di esempio}, la potenza più grande minore o uguale a 32 è $32 = 2^5$.
