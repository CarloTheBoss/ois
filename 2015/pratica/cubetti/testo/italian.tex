\usepackage{xcolor}
\usepackage{afterpage}

Grazie alle sue grandi abilità di cecchinaggio\footnote{Nel gergo delle aste online, con \emph{cecchinaggio} ci si riferisce all'atto di fare offerte all'ultimo secondo.}, Giorgio si è aggiudicato una fantastica collezione di preziosissimi cubetti colorati da collezione presso un'asta online. Purtroppo l'entusiasmo per il ricco bottino è scemato di colpo quando aprendo il cofanetto in pelle di porcospino ha scoperto che i cubetti non sono tutti di colori diversi come si aspettava. Come tutti i collezionisti di cubetti colorati sanno, una collezione di cubetti è di valore solo se i cubetti sono tutti di colori diversi. Per fortuna non è impreparato e per emergenze di questo tipo può fare affidamento alla sua fedele Vernici-o-matic\texttrademark, una verniciatrice ad alta precisione per cubetti colorati. La macchina funziona in modo molto semplice: si inserisce un cubetto colorato, si imposta il nuovo colore e il cubetto viene verniciato di quel colore. Dal momento che il processo di verniciatura è molto lento e che Giorgio non vede l'ora di mostrare ai suoi amici la nuova collezione, scrivi un programma che determini il numero minimo di cubetti da verniciare affinché alla fine tutti i cubetti abbiano colori diversi.

\InputFile
La prima riga del file di input contiene l'intero $N$, il numero di cubetti colorati all'interno del cofanetto. La seconda riga contiene $N$ interi $a_1, \ldots, a_N$ che rappresentano i colori dei diversi cubetti. In particolare vale sempre $1 \le a_i \le N$ per ogni $i=1,\ldots,N$ e numeri uguali corrispondono a colori uguali e viceversa.

\OutputFile
In output, stampare il numero minimo di cubetti da verniciare per ottenere cubetti di colori diversi.

\Implementation
Dovrai sottoporre esattamente un file con estensione \texttt{.c}, \texttt{.cpp} o \texttt{.pas}.

\begin{warning}
Tra gli allegati a questo task troverai un template (\texttt{cubetti.c}, \texttt{cubetti.cpp}, \texttt{cubetti.pas}) con un esempio di implementazione da completare.
\end{warning}

Se sceglierai di utilizzare il template, dovrai implementare la seguente funzione:
\begin{center}\begin{tabularx}{\textwidth}{|c|X|}
\hline
C/C++  & \verb|int diversifica(int N, int colore[]);|\\
\hline
Pascal & \small\verb|function diversifica(N: longint, var colore: array of longint): longint;|\\
\hline
\end{tabularx}\end{center}
La funzione riceverà come parametro il numero di cubetti $N$ e l'array dei colori dei cubetti, e dovrà ritornare la risposta al problema, che verrà stampata sul file di output.

\Constraints 
\begin{itemize}[nolistsep,itemsep=2mm]
  \item $2 \le N \le 100\,000$.
  \item La verniciatrice può applicare una vernice di colore arbitrario (da 1 a $N$) ad ogni cubetto.
  \item L'input contiene almeno due cubetti di colore uguale.
\end{itemize}

\Scoring
Il tuo programma verrà testato su diversi test case raggruppati in subtask.
Per ottenere il punteggio relativo ad un subtask, è necessario risolvere
correttamente tutti i test relativi ad esso.

\begin{itemize}[nolistsep,itemsep=2mm]
  \item \textbf{\makebox[2cm][l]{Subtask 1} [10 punti]}: Caso d'esempio.
  \item \textbf{\makebox[2cm][l]{Subtask 2} [20 punti]}: $N \le 100$.
  \item \textbf{\makebox[2cm][l]{Subtask 3} [40 punti]}: $N \le 10\,000$.
  \item \textbf{\makebox[2cm][l]{Subtask 4} [30 punti]}: Nessuna limitazione specifica.
\end{itemize}

\Examples
\begin{example}
\exmp{
5
5 1 2 2 2
}{%
2
}%
\end{example}

\Explanation
Nel caso di esempio è sufficiente verniciare il terzo elemento del colore 3 e il quinto elemento del colore 4.