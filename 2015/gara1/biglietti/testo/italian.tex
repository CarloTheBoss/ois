% \usepackage{xcolor}
% \usepackage{afterpage}
% \usepackage{pifont,mdframed}
% \usepackage[bottom]{footmisc}

\makeatletter
\gdef\this@inputfilename{input.txt}
\gdef\this@outputfilename{output.txt}
\makeatother

\createsection{\Grader}{Grader di prova}

\newcommand{\inputfile}{\texttt{input.txt}}
\newcommand{\outputfile}{\texttt{output.txt}}

\newenvironment{warning}
  {\par\begin{mdframed}[linewidth=2pt,linecolor=gray]%
    \begin{list}{}{\leftmargin=1cm
                   \labelwidth=\leftmargin}\item[\Large\ding{43}]}
  {\end{list}\end{mdframed}\par}

Giorgio, che studia a Torino, ha deciso di far visita a Gabriele, che studia a Milano. Giorgio sa che durante il periodo che passerà con Gabriele avrà bisogno di fare $N$ viaggi sui mezzi pubblici, e per questo sta indagando sui prezzi dei biglietti. Ha scoperto che a Milano è possibile comprare un biglietto valido per una singola corsa per $A$ centesimi, oppure un carnet da $M$ viaggi, al costo di $B$ centesimi.

Conoscendo $N$, $M$, $A$ e $B$, quanti centesimi al minimo deve spendere Giorgio per poter fare $N$ corse sui mezzi?

\InputFile
Il file \inputfile{} è composto da un unica riga contenente gli interi $N, M, A, B$.

\OutputFile
Il file \outputfile{} è composto da un'unica riga contenente un unico intero, la risposta a questo problema.

\Implementation
Dovrai sottoporre esattamente un file con estensione \texttt{.c}, \texttt{.cpp} o \texttt{.pas}.

\begin{warning}
Tra gli allegati a questo task troverai un template (\texttt{biglietti.c}, \texttt{biglietti.cpp}, \texttt{biglietti.pas}) con un esempio di implementazione.
\end{warning}

Se sceglierai di utilizzare il template, dovrai implementare la seguente funzione:
\begin{center}\begin{tabularx}{\textwidth}{|c|X|}
\hline
C/C++  & \verb|int compra(int N, int M, int A, int B);|\\
\hline
Pascal & \verb|function compra(N, M, A, B: longint): longint;|\\
\hline
\end{tabularx}\end{center}
In cui:
\begin{itemize}[nolistsep]
  \item L'intero $N$ rappresenta il numero di corse che Giorgio deve fare.
  \item L'intero $M$ rappresenta il numero di corse che sono comprese in un carnet.
  \item L'intero $A$ rappresenta il costo in centesimi di una corsa singola.
  \item L'intero $B$ rappresenta il costo in centesimi di un intero carnet.
  \item La funzione dovrà restituire il minimo numero di centesimi che è necessario spendere, che verrà stampato sul file di output.
\end{itemize}

\Constraints 
\begin{itemize}[nolistsep,itemsep=2mm]
  \item $1 \le N, M, A, B \le 10\,000$.
  \item È possibile che Giorgio compri un numero di corse maggiore di $N$, se conveniente.
\end{itemize}

\newpage
\Scoring
Il tuo programma verrà testato su diversi test case raggruppati in subtask.
Per ottenere il punteggio relativo ad un subtask, è necessario risolvere
correttamente tutti i test relativi ad esso.

\begin{itemize}[nolistsep,itemsep=2mm]
  \item \textbf{\makebox[2cm][l]{Subtask 1} [10 punti]}: Casi d'esempio.
  \item \textbf{\makebox[2cm][l]{Subtask 2} [20 punti]}: $1 \le N \le 100$.
  \item \textbf{\makebox[2cm][l]{Subtask 3} [40 punti]}: $1 \le N \le 1000$.
  \item \textbf{\makebox[2cm][l]{Subtask 4} [30 punti]}: Nessuna limitazione specifica.
\end{itemize}

\Examples
\begin{example}
\exmp{
4 10 150 1380
}{%
600
}%
\end{example}
\begin{example}
\exmp{
11 10 150 1380
}{%
1530
}%
\end{example}
\begin{example}
\exmp{
10 10 150 1700
}{%
1500
}%
\end{example}
\begin{example}
\exmp{
11 10 150 100
}{%
200
}%
\end{example}


\Explanation
Nel \textbf{primo caso di esempio} conviene comprare 4 biglietti singoli.\\[2mm]
Nel \textbf{secondo caso di esempio} conviene comprare un carnet e un biglietto.\\[2mm]
Nel \textbf{terzo caso di esempio} conviene comprare 10 biglietti singoli.\\[2mm]
Nel \textbf{quarto caso di esempio} conviene comprare due carnet.
