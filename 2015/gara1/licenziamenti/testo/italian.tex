\usepackage{xcolor}
\usepackage{afterpage}
\usepackage{pifont,mdframed}
\usepackage[bottom]{footmisc}

\makeatletter
\gdef\this@inputfilename{input.txt}
\gdef\this@outputfilename{output.txt}
\makeatother

\newcommand{\inputfile}{\texttt{input.txt}}
\newcommand{\outputfile}{\texttt{output.txt}}

\newenvironment{warning}
  {\par\begin{mdframed}[linewidth=2pt,linecolor=gray]%
    \begin{list}{}{\leftmargin=1cm
                   \labelwidth=\leftmargin}\item[\Large\ding{43}]}
  {\end{list}\end{mdframed}\par}

	La \emph{SteamPower S.P.A.}, azienda leader mondiale nel campo delle macchine a vapore portatili, sta attraversando un brutto periodo di crisi e per tenerla in piedi nell'attesa che l'economia riparta la dirigenza ha stabilito un regime di forti tagli al personale.

	Per stabilire chi deve essere licenziato, il reparto personale ha distribuito un test con cui ha stabilito accuratamente di ognuno degli $N$ dipendenti il suo personale valore di bravura $B_i$. Inoltre si baserà anche sullo studio dell'organigramma (cioè il grafico ad albero in cui tutti i dipendenti dell'azienda sono organizzati secondo le relazioni di dipendenza), e conosce quindi il capo diretto $C_i$ di ogni dipendente $i$. Per convenzione, il presidente dell'azienda è segnato al numero $0$ e ha $C_i = -1$ per indicare l'assenza di un capo diretto.

	Le direttive presidenziali hanno stabilito che per cominciare verranno licenziati tutti i dipendenti \emph{inadeguati}, e cioè quelli che hanno valore di bravura $B_i$ strettamente maggiore di quello di uno dei loro capi (diretti o indiretti). Aiuta il reparto personale a determinare quante persone dovranno essere licenziate!

\InputFile
Il file \inputfile{} è composto da $N+1$ righe. La prima riga contiene l'unico intero $N$. Le successive $N$ righe contengono ciascuna due interi separati da uno spazio, i valori $B_i$ e $C_i$ del dipendente $i$-esimo.

\OutputFile
Il file \outputfile{} è composto da un'unica riga contenente un unico intero, la risposta a questo problema.

\Implementation
Dovrai sottoporre esattamente un file con estensione \texttt{.c}, \texttt{.cpp} o \texttt{.pas}.

\begin{warning}
Tra gli allegati a questo task troverai un template (\texttt{licenziamenti.c}, \texttt{licenziamenti.cpp}, \texttt{licenziamenti.pas}) con un esempio di implementazione da completare.
\end{warning}

Se sceglierai di utilizzare il template, dovrai implementare la seguente funzione:
\begin{center}\begin{tabularx}{\textwidth}{|c|X|}
\hline
C/C++  & \verb|int licenzia(int N, int B[], int C[]);|\\
\hline
Pascal & \verb|function licenzia(N: longint; var B, C: array of longint): longint;|\\
\hline
\end{tabularx}\end{center}
In cui:
\begin{itemize}[nolistsep]
  \item L'intero $N$ rappresenta il numero totale di dipendenti dell'azienda.
  \item L'array \texttt{B}, indicizzato da $0$ a $N-1$, contiene i valori di bravura $B_i$ dei dipendenti.
  \item L'array \texttt{C}, indicizzato da $0$ a $N-1$, contiene gli indici dei capi diretti $C_i$ (sempre da 0 a $N-1$) dei dipendenti corrispondenti. Il valore $C_0$ è garantito essere pari a $-1$ e indica che il presidente è sempre il dipendente numero $0$.
  \item La funzione dovrà restituire il numero di dipendenti inadeguati, che verrà stampato sul file di output.
\end{itemize}

% Assunzioni
\Constraints
\begin{itemize}[nolistsep, itemsep=2mm]
	\item $1 \le N \le 100\,000$.
	\item $0 \le B_i \le 1\,000\,000$ per ogni $i=0\ldots N-1$.
	\item $0 \le C_i \le N-1$ per ogni $i=1\ldots N-1$, mentre $C_0 = -1$.
	\item \`E garantito che l'organigramma rappresenti effettivamente un albero, e cioè che risalendo di capo in capo a partire da ogni dipendente $i$ si arrivi sempre al presidente $0$.
\end{itemize}

\Scoring
Il tuo programma verrà testato su diversi test case raggruppati in subtask.
Per ottenere il punteggio relativo ad un subtask, è necessario risolvere
correttamente tutti i test relativi ad esso.

\begin{itemize}[nolistsep,itemsep=2mm]
  \item \textbf{\makebox[2cm][l]{Subtask 1} [10 punti]}: Casi d'esempio.
  \item \textbf{\makebox[2cm][l]{Subtask 2} [20 punti]}: $N \leq 10$.
  \item \textbf{\makebox[2cm][l]{Subtask 3} [30 punti]}: $N \leq 100$.
  \item \textbf{\makebox[2cm][l]{Subtask 4} [10 punti]}: L'organigramma forma una linea retta, e quindi $C_i = i-1$ per ogni $i$.
  \item \textbf{\makebox[2cm][l]{Subtask 5} [30 punti]}: Nessuna limitazione specifica.
\end{itemize}

% Esempi
\Examples
\begin{example}
\exmp{
4
5 -1
2 0
8 0
4 0
}{%
1
}%
\end{example}
\begin{example}
\exmp{
7
4 -1
2 0
1 4
2 4
6 0
3 1
5 4
}{%
3
}%
\end{example}


\Explanation
Il \textbf{primo caso di esempio} corrisponde a questo organigramma, in cui i numeri bianchi nei riquadri scuri rappresentano il valore di bravura dei dipendenti:
\begin{figure}[H]%
\centering\includegraphics{asy_licenziamenti/fig1.pdf}%
\end{figure}
Il dipendente 2 ha un valore di bravura superiore a quello del suo capo (il presidente dell'azienda), e per questo deve essere licenziato.\\[2mm]
Il \textbf{secondo caso di esempio} corrisponde invece a questo organigramma:
\begin{figure}[H]%
\centering\includegraphics{asy_licenziamenti/fig2.pdf}%
\end{figure}
I dipendenti 4 e 5 hanno un valore di bravura maggiore dei loro rispettivi capi, e quindi vanno licenziati. Il dipendente 6 ha un valore di bravura minore di quello del suo capo, ma maggiore di quello del capo del suo capo, e quindi va licenziato.