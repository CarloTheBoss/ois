% \usepackage{xcolor}
% \usepackage{afterpage}
% \usepackage{pifont,mdframed}
% \usepackage[bottom]{footmisc}

\makeatletter
\gdef\this@inputfilename{input.txt}
\gdef\this@outputfilename{output.txt}
\makeatother

\newcommand{\inputfile}{\texttt{input.txt}}
\newcommand{\outputfile}{\texttt{output.txt}}

\newenvironment{warning}
  {\par\begin{mdframed}[linewidth=2pt,linecolor=gray]%
    \begin{list}{}{\leftmargin=1cm
                   \labelwidth=\leftmargin}\item[\Large\ding{43}]}
  {\end{list}\end{mdframed}\par}

Dopo il successo dello spettacolo con le piroette, Giorgio si è assicurato una brillante carriera nel mondo della coreografia. Per il prossimo spettacolo Giorgio sta pensando a qualcosa di decisamente più audace e dinamico: una lunghissima fila di trampolini elastici, ognuno a un metro di distanza dal precedente. Al termine della fila di trampolini è posto un tappetone elastico.

Ogni trampolino elastico è dotato di una elasticità $E$, che rappresenta il numero massimo di metri di lunghezza che è possibile compiere con un salto su quel trampolino. Ad esempio, se $E = 1$, l'acrobata dopo un balzo può trovarsi solo al trampolino successivo, mentre se il trampolino corrente ha $E = 3$ l'atleta può dosare la forza del salto e trovarsi in uno dei 3 trampolini successivi al corrente.

Data la sequenza dei trampolini e delle loro elasticità, aiuta Giorgio a determinare quale è il minimo numero di salti che è necessario che compiano gli acrobati per terminare sul tappetone, sapendo che il primo balzo avviene obbligatoriamente sul primo trampolino.

\Implementation

\begin{warning}
Tra gli allegati a questo task troverai un template (\texttt{trampolino.c}, \texttt{trampolino.cpp}, \texttt{trampolino.pas}) con un esempio di implementazione da completare.
\end{warning}

Se sceglierai di utilizzare il template, dovrai implementare la seguente funzione:
\begin{center}\begin{tabularx}{\textwidth}{|c|X|}
\hline
C/C++  & \verb|int salta(int N, int E[]);|\\
\hline
Pascal & \verb|function salta(N: longint; var E: array of longint): longint;|\\
\hline
\end{tabularx}\end{center}
In cui:
\begin{itemize}[nolistsep]
  \item L'intero $N$ rappresenta il numero di trampolini presenti in scena.
  \item L'array \texttt{E}, indicizzato da $0$ a $N-1$, contiene l'elasticità dei trampolini.
  \item La funzione dovrà restituire il minimo numero di salti necessari per finire sul tappetone, che verrà stampato sul file di output.
\end{itemize}

\InputFile
Il file \inputfile{} è composto da due righe. La prima riga contiene l'unico intero $N$. La seconda riga contiene gli $N$ interi $E_i$ separati da uno spazio.

\OutputFile
Il file \outputfile{} è composto da un'unica riga contenente un unico intero, la risposta a questo problema.

\pagebreak
% Assunzioni
\Constraints
\begin{itemize}[nolistsep, itemsep=2mm]
	\item $1 \le N \le 100\,000$.
	\item $1 \le E_i \le 100\,000$ per ogni $i=0\ldots N-1$.
	\item Non è possibile saltare all'indietro.
	\item È obbligatorio che il primo salto avvenga sul primo trampolino.
\end{itemize}

\Scoring
Il tuo programma verrà testato su diversi test case raggruppati in subtask.
Per ottenere il punteggio relativo ad un subtask, è necessario risolvere
correttamente tutti i test relativi ad esso.

\begin{itemize}[nolistsep,itemsep=2mm]
  \item \textbf{\makebox[2cm][l]{Subtask 1} [10 punti]}: Casi d'esempio.
  \item \textbf{\makebox[2cm][l]{Subtask 2} [20 punti]}: $N \leq 100$.
  \item \textbf{\makebox[2cm][l]{Subtask 3} [40 punti]}: $N \leq 1000$.
  \item \textbf{\makebox[2cm][l]{Subtask 4} [30 punti]}: Nessuna limitazione specifica.
\end{itemize}

% Esempi
\Examples
\begin{example}
\exmp{
4
2 3 1 1
}{%
2
}%
\end{example}
\begin{example}
\exmp{
5
5 2 3 4 5
}{%
1
}%
\end{example}
\begin{example}
\exmp{
8
4 2 3 1 1 2 1 2
}{%
4
}%
\end{example}


\Explanation
Nel \textbf{primo caso di esempio} conviene dosare il salto sul primo trampolino in modo da arrivare al secondo trampolino e da qui arrivare al tappetone. Saltare dal primo trampolino al terzo sarebbe costato 3 salti invece di 2.\\[2mm]
Nel \textbf{secondo caso di esempio} gli atleti saltano dal primo trampolino direttamente sul tappetone.\\[2mm]
Il \textbf{terzo caso di esempio} corrisponde alla figura. Il colore del centro dei trampolini ha intensità proporzionale all'elasticità.
\begin{center}
	\centering\includegraphics[scale=.8]{asy_trampolino/es3.pdf}
\end{center}
